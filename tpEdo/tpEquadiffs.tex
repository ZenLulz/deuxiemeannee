\documentclass[a4paper,10pt]{article}
\usepackage[utf8]{inputenc}
\usepackage[french]{babel}
\usepackage{amsfonts,bm,amsmath,amssymb,graphicx}
\usepackage{cancel}

\newcommand{\dd}{\mathrm{d}}
\newcommand{\real}{\mathbb{R}}
\newcommand{\integer}{\mathbb{Z}}
\newcommand{\definition}{\textbf{Definition }}
\newcommand{\exemples}{\textbf{Exemples }}
\newcommand{\remarque}{\textbf{Remarque }}
\newcommand{\proprietes}{\textbf{Propriétés }}
\newcommand{\propriete}{\textbf{Propriété }}

\title{Travaux pratiques: \'Equations différentielles ordinaires}
% \author{Orestis Malaspinas}
\date{}

\begin{document}
\maketitle

\section*{Travaux pratiques: Équations différentielles}

\subsection*{Rappel théorique : approximation numérique d’équations
différentielles ordinaires}

En général il n’existe pas de solutions analytiques pour les équations différentielles d’un intérêt pratique en ingénierie, 
c’est pourquoi on est obligé d’utiliser des méthodes
numériques pour approximer la solution.
Nous présentons ici une famille de méthodes numériques : les méthodes de
Runge-Kutta qui permettent d’approximer
les solutions d’équations différentielles ordinaires
\begin{equation*}
 \frac{\dd y}{\dd t} = F(y,t), \quad y(t_0)=y_0,
\end{equation*}
où $F$ est une fonction qui dépend de $y$ et de $t$.

\subsection*{Les méthodes de Runge-Kutta}

Les méthodes de Runge--Kutta sont une famille de méthodes explicites pour résoudre des équations 
différentielles ordinaires. Elles ont la forme général suivante
\begin{equation}
 y_{n+1}=y_0+\delta t\sum_{i=1}^s b_ik_i,
\end{equation}
où $\delta t$ est le pas de temps, $y_n\equiv y(t_n)$ ($t_n=t_0+n\delta t$). La valeur des 
$k_i$ dépend de la précision de l'algorithme (et donc de la valeur du nombre ``d'étages'' $s$) et ils sont donnés
de façon itérative par
\begin{align*}
 k_1&=F(y_n,t_n),\\
 k_2&=F(y_n+a_{21}\delta t k_1,t_n+c_2\delta t),\\
 &\cdots\\
 k_s&=F(y_n+a_{s1}\delta t k_1+a_{s2}\delta t k_2+\cdots +a_{s,s-1}\delta t k_{s-1},t_n+c_s \delta t).
\end{align*}
et où les $a_{ij}$, $b_i$ et $c_i$ sont propres de nouveaux au nombre d'étages $s$ de la méthode.
Pour $s=1$, on a simplement $c_1=0$, $b_1=1$ et donc il vient
\begin{equation*}
 y_{n+1}=y_n+\delta t F(y_n,t_n).
\end{equation*}
Cette méthode s'appelle également méthode d'Euler explicite (vous l'avez peut-être reconnue). 
Pour $s=2$, on obtient $c_1=0$, $c_2=1/2$, $a_{21}=1/2$, $b_1=0$, et $b_2=1$. On a donc
\begin{equation*}
y_{n+1}=y_n+\delta t F\left(y_n+\frac{\delta t}{2}F(y_n,t_n),t_n+\frac{\delta t}{2}\right).
\end{equation*}
Cette méthode s'appelle également méthode de point du milieu (elle est proche de la méthode du point du milieu pour les intégrales).

\section*{Équation de Lorenz}

L'équation de Lorenz est un système d'équations différentielles ordinaires agissant sur une variable
vectorielle à trois composantes $y(t)=(y_x(t),y_y(t),y_z(t))$
\begin{align*}
 \frac{\dd y_x}{\dd t}&=\sigma(y_y-y_x),\\
 \frac{\dd y_y}{\dd t}&=y_x(\rho-y_z)-y_y,\\
 \frac{\dd y_z}{\dd t}&=y_yy_x-\beta y_z,
\end{align*}
où les constantes sont données par $\beta=8/3$, $\sigma=10$, et $\rho=28$.
\begin{figure}[htp]
\begin{center}
\includegraphics[width=0.5\textwidth]{../figs/1024px-Lorenz_system_r28_s10_b2-6666.png}
\caption{Solution dans le plan $x-z$ de l'équation de Lorenz pour une solution initiale donnée (image tirée de Wikipedia).}\label{fig_lorenz}
\end{center}
\end{figure}
La solution de l'équation de Lorenz est dite \textit{chaotique}. Cela signifie que
si nous considérons deux conditions initiales $y^1$ et $y^2$ qui sont très proches. 
Elles s'éloigneront l'une de l'autre exponentiellement vite au cours du temps.

Par ailleurs, comme on peut le constater sur la figure \ref{fig_lorenz} la solution a une structure de ``papillon''. 
Cette structure est retrouvée indépendemment de la condition initiale (on aura toujours cette forme de solution)
bien que la position à un temps donné ne soit pas la même pour deux conditions initiales différentes).

\subsection*{Calcul d'erreur}

L'erreur globale comise lors de l'intégration numérique d'une équation différentielle se calcule 
de la façon suivante
\begin{equation}
 E=\frac{1}{N}\sqrt{\sum_{i=0}^{N-1} E_i^2},
\end{equation}
où $N$ est le nombre de points dans l'interval d'intégration, 
et $E_i=y(t_n)-y_{n}$ est la différence entre la solution exacte au temps $t_n$ et la solution numérique 
de l'équation différentielle au temps $t_n$.


\section*{Travail à effectuer}

\subsection*{Écriture des solveurs}

Écrire un solveur pour la méthode d'Euler explicite (méthode de Runge--Kutta à
un étage) et un pour la méthode de Runge--Kutta à deux étages. Veillez à ce que ces solveurs 
marchent avec des grandeurs $y$ scalaires ou $n$-dimensionnelles. 

\subsection*{Validation des solveurs}
Valider ces solveurs sur une équation différentielle ``simple'', telle que
\begin{equation*}
 \frac{\dd y}{\dd t}=\nu\left(1-\frac{y}{C}\right)y,\quad y(t_0)=y_0.
\end{equation*}
dont la solution est donnée par 
\begin{equation*}
 y(t)=\frac{C}{1+\frac{C-y_0}{y_0}e^{-\nu t}}.
\end{equation*}
Choisir par exemple les paramètres suivants $y_0=10$ ($t_0=0$), $C=50$ et $\nu=1$. Au passage, cette équation 
peut représenter la croissance/décroissance en présence d'une limite maximale de population de $C$ individus.

\subsection*{Étude de convergence des solveurs}

Pour chacun des solveurs, étudier l'erreur en fonction du temps $t$ entre $t_0=0$ et $t_{max}=10$ pour $\delta t=0.1$. 
\begin{enumerate}
\item Faire un graphique 
pour la solution $y(t)$ obtenue avec les deux solveurs et la solution analytique en fonction du temps. 
\item Effectuer un second graphique avec de l'erreur en fonction du temps (l'erreur est la différence entre la valeur de la solution approximée par les solveurs et la solution analytique).
\item Finalement pour ces mêmes paramètres faire varier $\delta t$ ($\delta t=0.1,0.05, 0.025, 0.0125$). Pour chacune de ces valeurs de $\delta t$ calculer l'erreur à $t=t_{max}$
et faire un graphique de l'erreur obtenue en fonction de $\delta t$ pour chacun des deux solveurs (les deux résultats sur le même graphique).
Ce graphique doit être fait en échelle log-log. Que pouvez-vous dire sur l'ordre de l'erreur?
\end{enumerate}

\subsection*{Équation de Lorenz}

Appliquer les deux solveurs à l'équation de Lorenz et reproduire une figure approchant celle de l'énoncé (faire un graphique éventuellement tri-dimmensionnel de la trajectoire obtenue
pour une condition initiale quelconque).

\subsection*{Orbite périodiques}

L’attracteur de Lorenz contient des trajectoires périodiques sur lesquelles
le système revient au point initial après un certain temps. Ces trajectoires sont dites
``instables'', comme toute autre trajectoire de ce système : il suffit qu’on dévie un
tout petit peu de la trajectoire, et on finit par s’en éloigner complètement. Le
point suivant se trouve (dans la limite de la précision numérique fournie) sur
une trajectoire périodique de période $t_{max}=1.5586522$:
$y_0=(-0.9101673912,-1.922121396,18.18952097)$.
Prendre ce point comme valeur initiale, et faites évoluer le système avec un pas de
temps donné (par exemple $\delta t = 0.001$). Combien de temps arrivez-vous à rester
sur l’orbite périodique avec chacun des solveurs?
Tracez les deux trajectoires sur une figure 3D pour les comparer.

\section*{Remarques}

Le travail peut-être effectué en groupe de deux ou seul
(n'oubliez pas de mentionner les deux noms sur le rapport et dans le code si le travail est fait à deux). 
Je dois pouvoir exécuter le code
afin de pouvoir reproduire les résultats présentés dans le rapport. Je dois aussi pouvoir 
définir ma propre fonction à intégrer de façon simple.
Le rapport (en PDF!) et le code doivent être déposés sur cyberlearn. De préférence déposez une archive avec le code et
le pdf tel quel sur cyberlearn.

La note sera une combinaison entre le code rendu et le rapport (moitié/moitié). 



\end{document}
